%%%% Paramétrage du TD %%%%
\def\xxactivite{Colle \ifprof -- Corrigé \else \fi} % \normalsize \vspace{-.4cm}
\def\xxauteur{\textsl{Xavier Pessoles}}


\def\xxnumchapitre{Chapitre 1 \vspace{.2cm}}
\def\xxchapitre{\hspace{.12cm} }


\def\xxcompetences{%
\vspace{-.3cm}
\textsl{%
\textbf{Savoirs et compétences :}\\
\vspace{-.4cm}
%\begin{itemize}[label=\ding{112},font=\color{ocre}] 
%%\item \textit{Res1.C4 : } Correction
% \item \textit{Res1.C4.SF1 : } Proposer la démarche de réglage d’un correcteur proportionnel
%%proportionnel intégral 
%%et à avance de phase
%\item \textit{Con.C2 : } 	Correction d’un système asservi	
%\item \textit{Con.C2.SF1 : } Choisir un type de correcteur adapté
%\end{itemize}
}}

\def\xxauteur{\textsl{Xavier Pessoles}}

\def\xxtitreexo{Pompe hydraulique de bateau\ifnormal $\star$ \else \fi \iftdifficile $\star\star\star$ \else \fi }
\def\xxsourceexo{\hspace{.2cm} \footnotesize{}}

\def\xxfigures{
\includegraphics[width=.6\linewidth]{images/fig_01}
}%figues de la page de garde


\iflivret
\input{../../style/new_pagegarde}
\else
\input{../../style/new_pagegarde}
\fi
\setlength{\columnseprule}{.1pt}

\pagestyle{fancy}
\thispagestyle{plain}

\vspace{4.5cm}

\def\columnseprulecolor{\color{ocre}}
\setlength{\columnseprule}{0.4pt} 

\setcounter{exo}{0}

\ifprof
\begin{multicols}{2}
\else
\begin{multicols}{2}
\fi
%%%%%%%%%%%%%%%%%%%%%%%%%%%%%%%%%%%%%%%%%%%%%%%%%%


%\section{Système mobile d’imagerie interventionnelle Discovery IGS 730}
\subsection*{Fonctionnement}

Le dessin d'ensemble (voir le document format A4 à l’échelle 0,8 présent sur les documents-réponses) et la nomenclature fournie ci-après représente la pompe d'une barre à roue hydraulique installée sur un bateau de plaisance. La photo en en-tête du sujet représente une pompe similaire (à peu près identique) telle qu'on peut la trouver sur un catalogue de vente d'accessoires nautiques.


Cette pompe est reliée, par deux canalisations, à un vérin à double effet avec contre tige. La tige du vérin est reliée mécaniquement à la mèche (l’axe) du gouvernail (voir schéma d'implantation ci-après page 2).

La rotation de la barre à roue, dans un sens ou dans l'autre, provoque la sortie ou la rentrée de la tige du vérin, ce qui a pour effet de modifier l'orientation du safran (gouvernail) par rapport à l'axe du navire; par conséquent cela entraîne le virement du bateau dans la direction désirée.Il n’y pas de moteur électrique : c’est la rotation de la barre, grâce à l’action du barreur, qui seule fournit de l’énergie.

Le mécanisme étant réversible, deux clapets anti-retour pilotés (non représentés) sont intégrés dans le circuit hydraulique.


\subsection*{Description de la pompe}

\begin{center}
\includegraphics[width=\linewidth]{images/fig_02}
\end{center}

II s'agit d'une pompe à pistons axiaux, à cylindrée fixe.

Les pistons 4 s'appuient sur la rondelle de la butée 8 ; l'axe de cette butée étant incliné par rapport à l'axe de rotation du barillet  2.

Lors de la rotation du barillet 2 les pistons 4 sont animés, par rapport au barillet 2 d'un mouvement rectiligne alternatif. 

Si un piston se déplace suivant $\vect{i}$ (voir la coupe A-A) alors le cylindre correspondant est en phase d'Aspiration.

Si ce même piston se déplace suivant ($-\vect{i}$) alors le cylindre correspondant est en phase de Refoulement et le fluide hydraulique ainsi évacué alimente le récepteur.

Le corps de la pompe fait office de réservoir ; le fluide hydraulique assure la lubrification du mécanisme.


\subsection*{Travail à réaliser}
\subsection*{Analyse}
\subparagraph{}\textit{Définir la liaison entre le piston 4 et le barillet 2.}

\subparagraph{}\textit{Définir la liaison entre le piston 4 et la rondelle de la butée 8.}

\subparagraph{}\textit{Quelle est la fonction des ressorts 5 ?}

\subparagraph{}\textit{Dans la situation la plus défavorable, cas du piston apparent dans la coupe A-A, quel doit être l’état du ressort correspondant ?}




\subsection*{Fonctionnement de la barre à roue}

Les croquis 1 et 2 du document réponse 1 représentent :
\begin{itemize}
\item le safran
\item le vérin double effet à contre-tige
\item le distributeur 3 et le barillet 2 de la pompe
\item les canalisations reliant le distributeur 3 au vérin.
\end{itemize}

Au croquis 1 correspond le virement à tribord (vers la droite) du bateau (le safran tournant dans le sens trigonométrique autour de l’axe de la mèche).

Au croquis 2 correspond le virement à bâbord (vers la gauche) du bateau (le safran tournant dans le sens horaire autour de l’axe de la mèche).



\subparagraph{}\textit{Pour chacun des deux cas de fonctionnement, représenter dans la partie adéquate du circuit hydraulique,
\begin{itemize}
\item en rouge le fluide hydraulique moteur, c’est à dire au refoulement par rapport à la pompe
\item en bleu le fluide hydraulique passif, c’est à dire l’aspiration par rapport à la pompe. 
\end{itemize}}

\subparagraph{}\textit{Représenter, dans chaque zone, le sens de circulation du fluide hydraulique par des flèches.}

\subparagraph{}\textit{Indiquer dans les deux cas le sens de rotation du barillet, donc de la barre à roue.}

\subsection*{Schéma cinématique}
\subparagraph{}\textit{Mettre en place sur le dessin d’ensemble les sous-ensembles cinématiques par coloriage.}

\subparagraph{}\textit{Faire un graphe de structure.}

\subparagraph{}\textit{Faire le schéma cinématique (le ressort 5 ne sera pas représenté).}

La pièce 3 ayant une forme un délicate à appréhender le dessin de définition de cette pièce est donné ci-dessous.

\begin{center}
\includegraphics[width=\linewidth]{images/fig_03}
\end{center}




%%%%%%%%%%%%%%%%%%%%%%%%%%%%%%%%%%%%%%%%%%%%%%%%%%%

\ifprof
\end{multicols}
\else
\end{multicols}
\fi



\begin{center}
\includegraphics[width=\linewidth]{images/fig_04}
\end{center}