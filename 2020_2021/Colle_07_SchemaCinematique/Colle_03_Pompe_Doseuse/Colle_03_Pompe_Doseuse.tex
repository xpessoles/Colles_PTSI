%%%% Paramétrage du TD %%%%
\def\xxactivite{Colle \ifprof -- Corrigé \else \fi} % \normalsize \vspace{-.4cm}
\def\xxauteur{\textsl{Xavier Pessoles}}


\def\xxnumchapitre{Chapitre 1 \vspace{.2cm}}
\def\xxchapitre{\hspace{.12cm} }


\def\xxcompetences{%
\vspace{-.3cm}
\textsl{%
\textbf{Savoirs et compétences :}\\
\vspace{-.4cm}
%\begin{itemize}[label=\ding{112},font=\color{ocre}] 
%%\item \textit{Res1.C4 : } Correction
% \item \textit{Res1.C4.SF1 : } Proposer la démarche de réglage d’un correcteur proportionnel
%%proportionnel intégral 
%%et à avance de phase
%\item \textit{Con.C2 : } 	Correction d’un système asservi	
%\item \textit{Con.C2.SF1 : } Choisir un type de correcteur adapté
%\end{itemize}
}}

\def\xxauteur{\textsl{Xavier Pessoles}}

\def\xxtitreexo{Pompe doseuse\ifnormal $\star$ \else \fi \iftdifficile $\star\star\star$ \else \fi }
\def\xxsourceexo{\hspace{.2cm} \footnotesize{}}

\def\xxfigures{
%\includegraphics[width=.6\linewidth]{images/fig_01}
}%figues de la page de garde


\iflivret
\input{../../style/new_pagegarde}
\else
\input{../../style/new_pagegarde}
\fi
\setlength{\columnseprule}{.1pt}

\pagestyle{fancy}
\thispagestyle{plain}

\vspace{4.5cm}

\def\columnseprulecolor{\color{ocre}}
\setlength{\columnseprule}{0.4pt} 

\setcounter{exo}{0}

\ifprof
\begin{multicols}{2}
\else
\begin{multicols}{2}
\fi
%%%%%%%%%%%%%%%%%%%%%%%%%%%%%%%%%%%%%%%%%%%%%%%%%%



\subsection*{Mise en situation}
L’appareil à étudier est représenté sur le dessin d’ensemble (grande feuille A3). Il est associé à la nomenclature suivante. Le dessin est grosso-modo à l’échelle 0,35.

\begin{center}
\includegraphics[width=\linewidth]{images/fig_02}
\end{center}


Cette pompe doseuse est utilisée dans l’industrie agro-alimentaire pour effectuer le dosage de produits dont la consistance peut varier de l’état liquide (des soupes) à l’état très pâteux (des pâtés) avec des inclusions solides.

\begin{center}
\includegraphics[width=\linewidth]{images/fig_03}
\end{center}

Cet appareil peut être soit intégré dans une chaîne complète de conditionnement, soit utilisé de façon autonome.

Il est entraîné par un groupe de motorisation fixé sur 28, non représenté,  associant un moteur électrique, un variateur de vitesse à poulies et courroies, et un réducteur.

\begin{center}
\includegraphics[width=\linewidth]{images/fig_05}
\end{center}


\subsection*{Fonctionnement}
Le fonctionnement de la pompe est basé sur la combinaison :
\begin{itemize}
\item du mouvement de translation alternatif du piston (obtenu par un système bielle manivelle) : la bielle est constituée principalement des pièces 19 et 21.
\item du mouvement de rotation continu d’une chambre tournante 14 obtenu par un renvoi d’angle à pignons coniques (pignons 2 et 4).
\end{itemize}

\begin{center}
\includegraphics[width=\linewidth]{images/fig_06}
\end{center}

La chambre tournante comporte une lumière passant successivement devant l’orifice d’aspiration (lorsque le piston recule) et de refoulement (lorsque le piston avance).

\begin{center}
\includegraphics[width=\linewidth]{images/fig_07}
\end{center}




\subsection*{Régalge de la dose}
Le volume dosé est réglé en agissant sur l’excentration e du système bielle manivelle, au moyen du bouton moleté 23.

Ce bouton commande un système vis-écrou 24-25, dont le blocage en position après réglage est réalisé par serrage de l’écrou 31.


\begin{itemize}
\item Capacité de dosage : 0 à \SI{600}{cm^3}.
\item Cadence : environ 10 à 60 coups par minute suivant réglage du variateur.
\item Motorisation : moteur \SI{220}{V}/\SI{380}{V}, puissance 1, \SI{1}{KW}, fréquence de rotation \SI{1500}{tr/min}.
\end{itemize}
%%%%%%%%%%%%%%%%%%%%%%%%%%%%%%%%%%%%%%%%%%%%%%%%%%%
\subsection*{Étude technologique}

\subparagraph{}\textit{Le schéma ci-dessous décrit le mécanisme étudié dans une position quelconque pour un volume dosé de \SI{300}{cm^3}. Compléter le tracé du deuxième schéma décrivant le mécanisme étudié dans une position «piston en fin de phase de refoulement» pour un volume dosé de \SI{600}{cm^3}.}

\begin{center}
\includegraphics[width=\linewidth]{images/fig_08}
\end{center}


\subparagraph{}\textit{Justifier la présence d’un dispositif de réglage de longueur de la bielle (19+21).}

\subparagraph{}\textit{Décrire le mode opératoire à suivre pour procéder au réglage de la longueur de la bielle (19+21).}

\subparagraph{}\textit{Calculer, en utilisant la nomenclature le rapport des fréquences de rotation des roues coniques  2 et 4. Justifier en quoi cette valeur du rapport de transmission est impérative.}

\subparagraph{}\textit{L’étanchéité est assurée pour partie par des joints toriques. Du point de vue cinématique quelles sont les limites de leur utilisation.}

\subparagraph{}\textit{Déterminer par coloriage les sous-ensembles cinématiques directement sur le dessin d’ensemble. Il faudra le  joindre à la copie.}

\subparagraph{}\textit{Faire le schéma cinématique minimal. Les systèmes de réglage [réglage de la bielle et réglage de l'excentration] devront apparaître clairement sur votre schéma.}

\subsection*{Travail graphique}
\subparagraph{}\textit{Représenter la manivelle 3 :}
\begin{itemize}
\item vue de face extérieure (avec les arêtes cachées);
\item vue de gauche en coupe CC (sans arêtes cachées);
\item vue de dessus avec :
\begin{itemize}
\item demi vue extérieure (sans arêtes cachées),
\item demi coupe EE (sans arêtes cachées).
\end{itemize}
\end{itemize}
\ifprof
\end{multicols}
\else
\end{multicols}
\fi


\begin{center}
\includegraphics[width=\linewidth]{images/fig_09}
\end{center}