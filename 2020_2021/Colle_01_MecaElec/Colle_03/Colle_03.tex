\documentclass[10pt,fleqn]{article} % Default font size and left-justified equations
\usepackage[%
    pdftitle={Energétique},
    pdfauthor={Xavier Pessoles}]{hyperref}

    
\input{style/new_style}
\input{style/macros_SII}
\usepackage{multicol}
\usepackage{siunitx}
%\usepackage{picins}
\fichetrue
%\fichefalse

\proftrue

\proffalse

\tdtrue
%\tdfalse

\courstrue
\coursfalse


\def\classe{\textsf{PTSI}}
\def\xxnumpartie{}%Cycle --}
\def\xxpartie{ }

\def\xxnumchapitre{}%Chapitre -- \vspace{.2cm}}
\def\xxchapitre{\hspace{.12cm} }

\def\discipline{Sciences \\Industrielles de \\ l'Ingénieur}
\def\xxtete{Sciences Industrielles de l'Ingénieur}


  
\def\xxposongletx{2}
\def\xxposonglettext{1.45}
\def\xxposonglety{20}
%\def\xxonglet{Part. 1 -- Ch. 3}
\def\xxonglet{\textsf{}}%Cycle 05}}

\def\xxactivite{Colle 03}
\def\xxauteur{\textsl{Xavier Pessoles}}


\def\xxtitreexo{}%Siège motorisé}
\def\xxsourceexo{\hspace{.2cm} \footnotesize{}}%BTS CPI 2018}}


\def\xxcompetences{%
\vspace{-.5cm}
\footnotesize{
\textsl{%
\textbf{Savoirs et compétences :}\\
\vspace{-.2cm}
%\begin{itemize}[label=\ding{112},font=\color{ocre}] 
%\item Mod2.C18.SF1 : Déterminer l’énergie cinétique d’un solide, ou d’un ensemble de solides, dans son mouvement par rapport à un autre solide.
%\item Res1.C1.SF1 : Proposer une démarche permettant la détermination de la loi de mouvement.
%\item Mod1.C5.SF2 : Déterminer la puissance des actions mécaniques extérieures à un solide ou à un ensemble de solides, dans son mouvement rapport à un autre solide.
%\item Mod1.C5.SF3 : Déterminer la puissance des actions mécaniques intérieures à un ensemble de solides.
%\end{itemize}
}}}

\def\xxfigures{
\includegraphics[width=.5\textwidth]{images/fig_01}
}%figues de la page de garde


\def\xxpied{%
%Cycle 05 -- Modélisation mécanique -- Énergétique\\% afin de valider leurs performances.\\
%Chapitre 1 -- \xxactivite%
}

\setcounter{secnumdepth}{5}
%---------------------------------------------------------------------------


\begin{document}
%\chapterimage{png/Fond_Cin}
\input{style/new_pagegarde}
\vspace{5.5cm}
\pagestyle{fancy}
\thispagestyle{plain}


\def\columnseprulecolor{\color{ocre}}
\setlength{\columnseprule}{0.4pt} 

%\ifprof
%\else
\begin{multicols}{2}
%\fi
\section*{Exercice 1 -- Lois de Kirchoff}
\ifprof
\else
\fi


\subparagraph*{}
\textit{Sur le circuit suivant, déterminer les courants dans chacune des branches et la tension aux bornes de tous les dipôles en fonction de $E$ et des différentes résistances $R_i$.}
\begin{center}
\includegraphics[width=\linewidth]{images/fig_03}
\end{center}


\section*{Exercice 2 -- Résistance équivalente}
\textit{Déterminer la résistance équivalente du montage suivant.}
\begin{center}
\includegraphics[width=\linewidth]{images/fig_07}
\end{center}




\section*{Exercice 3 -- Mouvement de translation}

Joe Dupont conduit une voiture à $50\; \text{km}\,\text{h}^{-1}$ dans une rue horizontale. La voiture a une masse de $1\,060\; \text{kg}$. Soudain, il freine pour s’arrêter.  On suppose que la décélération est constante pendant tout le freinage ($a = -2\; \text{m}\,\text{s}^{-2}$).

\subparagraph{}
\textit{Indiquer la direction et le sens de la force exercée sur la voiture, calculer son intensité.}

\subparagraph{}
\textit{Calculer la durée du freinage.}

\subparagraph{}
\textit{Calculer la distance du freinage.}


\section*{Exercice 4 -- Calcul de moments}

\setcounter{subparagraph}{0}
On donne la structure suivante : 
\begin{center}
\includegraphics[width=.6\linewidth]{images/moment8}
\end{center}

\subparagraph{}
\textit{Déterminer $\vect{\mathcal{M}\left(A,\vect{R} \right)}$.}

On donne la structure suivante : 
\begin{center}
\includegraphics[width=.6\linewidth]{images/fig_25}
\end{center}


\subparagraph{}
\textit{Déterminer $\vect{\mathcal{M}\left(A,\vect{F} \right)}$ puis $\vect{\mathcal{M}\left(O,\vect{F} \right)}$.}



\end{multicols}

\end{document}

\subparagraph{}\textit{}
\ifprof
\begin{corrige}~\\
\end{corrige}
\else
\fi




\subparagraph{}\textit{}
\ifprof
\begin{corrige}~\\
\end{corrige}
\else
\fi

\subparagraph{}\textit{}
\ifprof
\begin{corrige}~\\
\end{corrige}
\else
\fi

\subparagraph{}\textit{}
\ifprof
\begin{corrige}~\\
\end{corrige}
\else
\fi

\subparagraph{}\textit{}
\ifprof
\begin{corrige}~\\
\end{corrige}
\else
\fi

\subparagraph{}\textit{}
\ifprof
\begin{corrige}~\\
\end{corrige}
\else
\fi

\subparagraph{}\textit{}
\ifprof
\begin{corrige}~\\
\end{corrige}
\else
\fi
\begin{center}
%\includegraphics[width=\linewidth]{images/fig_05}
\end{center}
