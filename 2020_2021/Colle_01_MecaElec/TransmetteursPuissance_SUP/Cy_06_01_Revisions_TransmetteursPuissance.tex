\documentclass[10pt,fleqn]{article} % Default font size and left-justified equations
\usepackage[%
    pdftitle={Modélisation dynamique},
    pdfauthor={Xavier Pessoles}]{hyperref}

    
\input{style/new_style}
\input{style/macros_SII}
\usepackage{multicol}
\usepackage{siunitx}
%\usepackage{picins}
\fichetrue
%\fichefalse

\proftrue
\proffalse

\tdtrue
%\tdfalse

\courstrue
\coursfalse


\def\classe{\textsf{PSI$\star$ -- MP}}
\def\xxnumpartie{Cycle 06}
\def\xxpartie{Modéliser le comportement des systèmes mécaniques dans le but d'établir une loi de comportement ou de déterminer des actions mécaniques en utilisant les méthodes énergétiques}

\def\xxnumchapitre{Chapitre 1 \vspace{.2cm}}
\def\xxchapitre{\hspace{.12cm} Approche énergétique}

\def\discipline{Sciences \\Industrielles de \\ l'Ingénieur}
\def\xxtete{Sciences Industrielles de l'Ingénieur}




\def\xxtitreexo{Révisions -- Rapports de transmission}%Motorisation du moteur Haibike}
\def\xxsourceexo{%\hspace{.2cm} \footnotesize{Émilien Durif -- E3A PSI 2011}
}


\def\xxposongletx{2}
\def\xxposonglettext{1.45}
\def\xxposonglety{20}
%\def\xxonglet{Part. 1 -- Ch. 3}
\def\xxonglet{\textsf{Cycle 06}}

\def\xxactivite{Application}
\def\xxauteur{\textsl{X. Pessoles}}

\def\xxcompetences{%
\textsl{%
\textbf{Savoirs et compétences :}\\
%Les sources sont associées par un \emph{hacheur série}. La détermination des grandeurs électriques associées à ce montage permet de conclure vis à vis du cahier des charges.
%\noindent \textbf{Résoudre :} à partir des modèles retenus :
%\begin{itemize}[label=\ding{112},font=\color{ocre}] 
%\item choisir une méthode de résolution analytique, graphique, numérique;
%\item mettre en \oe{}uvre une méthode de résolution.
%\end{itemize}
%\begin{itemize}[label=\ding{112},font=\color{ocre}] 
%\item \textit{Rés -- C1.1 :} Loi entrée sortie géométrique et cinématique -- Fermeture géométrique.
%\end{itemize}
%
%\noindent \textit{Mod2 -- C4.1 :} Représentation par schéma bloc.
}}

\def\xxfigures{
%\includegraphics[width=.7\linewidth]{images/axe_y_photo}
}%figues de la page de garde


\def\xxpied{%
Cycle 06 -- Modélisation mécanique -- Énergétique\\% afin de valider leurs performances.\\
Chapitre 1 -- \xxactivite%
}

\setcounter{secnumdepth}{5}
%---------------------------------------------------------------------------

\usepackage{pgfplots}
\begin{document}
\def\pathfig{images}
%\chapterimage{png/Fond_Cin}
\input{style/new_pagegarde}
\vspace{4.5cm}
\pagestyle{fancy}
\thispagestyle{plain}

\def\columnseprulecolor{\color{ocre}}
\setlength{\columnseprule}{0.4pt} 

\def\pathfig{images}

\ifprof
\begin{multicols}{2}
\else
\begin{multicols}{2}
\fi


\section*{Exercice 1 -- Train d'engrenages simple}
\setcounter{exo}{0}

\ifprof
\else
Soit le train d'engrenages suivant. 
\begin{center}
\includegraphics[width=.7\linewidth]{images/TrainSimple_01}
\end{center}
\fi

\subparagraph{}
\textit{Déterminer $\dfrac{\omega_{3/0}}{\omega_{1/0}}$ en fonction du nombre de dents des roues dentées.}
\ifprof
\begin{corrige}
On a $\dfrac{\omega_{3/0}}{\omega_{1/0}}=-\dfrac{Z_1}{Z_3}$.
\end{corrige}
\else
\fi

\subparagraph{}
\textit{Donner une relation géométrique entre $Z_1$, $Z_2$ et $Z_3$ permettant de garantir le fonctionnement du train d'engrenages. }
\ifprof
\begin{corrige}
On a $Z_3 = 2Z_2 + Z_1$.
\end{corrige}
\else
\fi



\ifprof
\else
Soit le train d'engrenages suivant. 
\begin{center}
\includegraphics[width=.7\linewidth]{images/TrainSimple_02}
\end{center}
\fi

\subparagraph{}
\textit{Déterminer $\dfrac{\omega_{4/0}}{\omega_{1/0}}$ en fonction du nombre de dents des roues dentées.}
\ifprof
\begin{corrige}
On a $\dfrac{\omega_{4/0}}{\omega_{1/0}}=-\dfrac{Z_1Z_{22}}{Z_4Z_{21}}$.
\end{corrige}
\else
\fi

\subparagraph{}
\textit{Donner une relation géométrique entre $Z_1$, $Z_{21}$, $Z_{22}$ et $Z_4$ permettant de garantir le fonctionnement du train d'engrenages. }
\ifprof
\begin{corrige}
On a $Z_1+Z_{21}+Z_{22}= Z_4$.
\end{corrige}
\else
\fi



\ifprof
\else
Soit le train d'engrenages suivant. 
\begin{center}
\includegraphics[width=.7\linewidth]{images/TrainSimple_03}
\end{center}
\fi


\subparagraph{}
\textit{Déterminer $\dfrac{\omega_{4/0}}{\omega_{1/0}}$ en fonction du nombre de dents des roues dentées.}
\ifprof
\begin{corrige}
On a $\dfrac{\omega_{4/0}}{\omega_{1/0}}=\dfrac{Z_1Z_{22}}{Z_4Z_{21}}$.
\end{corrige}
\else
\fi


\ifprof
\else
Soit le train d'engrenages suivant. 
\begin{center}
\includegraphics[width=.7\linewidth]{images/TrainSimple_04}
\end{center}
\fi


\subparagraph{}
\textit{Déterminer $\dfrac{\omega_{4/0}}{\omega_{1/0}}$ en fonction du nombre de dents des roues dentées.}
\ifprof
\begin{corrige}
On a $\dfrac{\omega_{4/0}}{\omega_{1/0}}=\dfrac{Z_1Z_{22}}{Z_4Z_{21}}$.
\end{corrige}
\else
\fi




\section*{Exercice 2 -- Train d'engrenages (cheville NAO)}
\setcounter{exo}{0}


\ifprof
\else
NAO est un robot humanoïde conçu par la société française Aldebaran. À l'origine il a été conçu comme prototype du robot Romeo, destiné à être au service des personnes. NAO est utilisé à l'heure actuelle dans la recherche en robotique et dans des domaines pédagogiques. 
\begin{obj}
On s'intéresse ici à la cheville NAO. On cherche à savoir si, à partir du moteur retenu par le constructeur, la chaîne de transmission de puissance permet de vérifier les exigences suivantes : 
\begin{itemize}
\item exigence 1.1.1.1 : la vitesse de roulis doit être inférieure à \SI{42}{tr/min};
\item exigence 1.1.1.2 : la vitesse de tangage doit être inférieure à \SI{60}{tr/min}.
\end{itemize}

\end{obj}



\begin{center}
\includegraphics[width=.7\linewidth]{images/nao_01}
\end{center}

La fréquence de rotation des moteurs permettant chacun des deux mouvements est de \SI{8300}{tr/min}.

Pour la chaîne de transmission de tangage on donne  le nombre de dents et le module de chaque roue dentée : 
\begin{itemize}
\item pignon moteur : $Z_m=20$, $M_m=0,3$;
\item grand pignon 1 : $Z_1 = 80$, $M_1=0,3$;
\item petit pignon 1 : $Z_1' = 25$, $M_1'=0,4$;
\item grand pignon 2 : $Z_2 = 47$, $M_2=0,4$;
\item petit pignon 2 : $Z_2' = 12$, $M_2'=0,4$;
\item grand pignon 3 : $Z_3 = 58$, $M_3=0,4$;
\item petit pignon 3 : $Z_3' = 10$, $M_3'=0,7$;
\item roue de sortie : $Z_3 = 36$, $M_3=0,7$.
\end{itemize}

\begin{center}
\includegraphics[width=.95\linewidth]{images/nao_02}
\end{center}

Pour la chaîne de transmission du roulis on donne le nombre de dents et le module de chaque roue dentée : 
\begin{itemize}
\item pignon moteur : $Z_m=13$, $M_m=0,3$;
\item grand pignon 1 : $Z_1 = 80$, $M_1=0,3$;
\item petit pignon 1 : $Z_1' = 25$, $M_1'=0,4$;
\item grand pignon 2 : $Z_2 = 47$, $M_2=0,4$;
\item petit pignon 2 : $Z_2' = 12$, $M_2'=0,4$;
\item grand pignon 3 : $Z_3 = 58$, $M_3=0,4$;
\item petit pignon 3 : $Z_3' = 10$, $M_3'=0,7$;
\item roue de sortie 3 : $Z_3 = 36$, $M_3=0,7$.
\end{itemize}



\begin{center}
\includegraphics[width=.8\linewidth]{images/nao_03}
\end{center}
\fi



\subparagraph{}
\textit{Quels doivent être les rapports de réductions des transmissions par engrenage afin de respecter les exigences 1.1.1.1 et 1.1.1.2 ?}
\ifprof
\begin{corrige}
D'après le diagramme de définition des blocs et le diagramme des exigences, les rapports de transmission doivent être : 
\begin{itemize}
\item pour l'axe de tangage : $\dfrac{N_{\text{moteur}}}{N_{\text{Tangage}}}=138,33$ au minimum; 
\item pour l'axe de roulis :  $\dfrac{N_{\text{moteur}}}{N_{\text{Roulis}}}= 197,61$ au minimum.
\end{itemize}
\end{corrige}
\else
\fi


\ifprof
\subparagraph{}
\textit{Dans le cas de l'axe de tangage, remplir le tableau suivant :}
\begin{corrige} ~\\
\begin{center}
\begin{tabular}{|p{1.7cm}|c|c|p{1.3cm}|}
\hline
Roue dentée & Module & Nb dents & Diamètre (mm)\\
\hline
Pignon 03 20 & 0,3 &20        & 6		\\ \hline
Mobile Inf1 Roue & 0,3 & 80  & 24 	\\ \hline 
Mobile Inf1 Pignon & 0,4 & 25 & 10	\\ \hline
Mobile Inf2 Roue & 0,4 & 47   & 18,8	\\ \hline
Mobile Inf2 Pignon & 0,4 & 12 & 4,8	\\ \hline
Mobile Inf4 Roue & 0,4 & 58   & 23,2	\\ \hline
Mobile Inf4 Pignon & 0,7 & 10 & 7	\\ \hline
Roue de sortie & 0,7 & 36      & 25,2 \\ \hline
\end{tabular}
\end{center}
\end{corrige}
\else
\fi

\subparagraph{}
\textit{Dans le cas de l'axe de tangage, déterminer le diamètre de chaque roue dentée.}
%
%\begin{center}
%\begin{tabular}{|l|c|c|c|}
%\hline
%Roue dentée & Module & Nb dents & Diamètre \\
%\hline
%& && \\ 
%Pignon 03 20 & && \\ 
%&& & \\ \hline
%&& & \\ 
%Mobile Inf1 Roue & && \\ 
%&& & \\ \hline
%&& & \\ 
%Mobile Inf1 Pignon & && \\ 
%&& & \\ \hline
%&& & \\ 
%Mobile Inf2 Roue & && \\ 
%&& & \\ \hline
%&& & \\ 
%Mobile Inf2 Pignon & && \\ 
%&& & \\ \hline
%&& & \\ 
%Mobile Inf4 Roue & && \\ 
%&& & \\ \hline
%&& & \\ 
%Mobile Inf4 Pignon & && \\ 
%&& & \\ \hline
%&& & \\ 
%Roue de sortie & && \\
%&& & \\ 
%\hline
%\end{tabular}
%\end{center}
%\fi
%
%
\subparagraph{}
\textit{Dans le cas de l'axe de tangage, réaliser le schéma cinématique minimal.}
\ifprof
\begin{corrige}
\end{corrige}
\else
\fi

\subparagraph{}
\textit{Calculer le rapport de transmission de la chaîne de transmission de l'axe de tangage ? L'exigence 1.1.1.2 est-elle respectée ? Si non, quelle(s) solution(s) de remédiation pourrait-on proposer ?}
\ifprof
\begin{corrige}
$$
R_T = (-1)^n \dfrac{80\cdot 47 \cdot 58 \cdot 36}{20\cdot 25\cdot 12 \cdot 10 } = 130,85
$$

Ceci est inférieur à ce qui est préconisé par le cahier des charges. 

Pour respecter le cahier des charges, on peut :
\begin{itemize}
\item choisir un autre moteur;
\item changer le nombre de dents d'une des roues. Il suffirait pour cela que,  par exemple, la roue de sortie comporte 39 dents. 
\end{itemize}
\end{corrige}
\else
\fi

\subparagraph{}
\textit{Calculer le rapport de transmission de la chaîne de transmission de l'axe de roulis ? L'exigence 1.1.1.1 est-elle respectée ? Si non, quelle(s) solution(s) de remédiation pourrait-on proposer ?}
\ifprof
\begin{corrige}
Le rapport de transmission du second train est de 201,3 ce qui est compatible avec le cahier des charges.
\end{corrige}
\else
\fi


\section*{Exercice 3 -- Réducteur de roue motrice de chariot élévateur}
\textit{D'après Florestan Mathurin.}
\setcounter{exo}{0}

\ifprof
\else

On s’intéresse au réducteur équipant la roue arrière motrice et directionnelle d’un chariot élévateur de manutention automoteur à conducteur non porté. 



\begin{center}
\includegraphics[width=.95\linewidth]{images/char_01}
\end{center}


\textbf{Données }: $z_{27} = \SI{16}{dents}$, $z_{35} = \SI{84}{dents}$, $z_{5} = \SI{14}{dents}$, $z_{11} = \SI{56}{dents}$, $z_{16} = \SI{75}{dents}$. 

\fi

\subparagraph{}
\textit{Identifier les classes d’équivalence cinématique sur le dessin d’ensemble. }
\ifprof
\begin{corrige}

\end{corrige}
\else
\begin{center}
\includegraphics[width=\linewidth]{images/char_02}
\end{center}

\fi



\subparagraph{}
\textit{ Construire le schéma cinématique du réducteur dans le même plan que le dessin.}
\ifprof
\begin{corrige}
\begin{center}
\includegraphics[width=\linewidth]{images/char_cor_01_BIS}
\end{center}
\end{corrige}
\else
\fi
\subparagraph{}
\textit{Compléter le tableau donnant les caractéristiques des roues et pignons.}
\ifprof
\begin{corrige}
\footnotesize
\begin{center}
\begin{tabular}{|c|c|c|c|}
\hline
Repère de  & Module  & Nombre & Diamètre primitif  \\
la roue & $m$ (mm) & de dents $Z$ & $D$ (mm) \\
\hline
\hline
27 & 1,5 &16 & 24\\ \hline
35 & 1,5 &84 & 126\\ \hline
5   &1,5 &14 & 21\\ \hline
11 & 1,5 & 56 & 84 \\ \hline
16 &  1,5&  75& 112,5\\ \hline

\end{tabular}
\end{center}

\normalsize
\end{corrige}
\else
\footnotesize
\begin{center}
\begin{tabular}{|c|c|c|c|}
\hline
Repère de  & Module  & Nombre & Diamètre primitif  \\
la roue & $m$ (mm) & de dents $Z$ & $D$ (mm) \\
\hline
\hline
27 & & & \\ \hline
35 & 1,5& & \\ \hline
5& & & \\ \hline
11& 1,5 & & \\ \hline
16& & & \\ \hline

\end{tabular}
\end{center}

\normalsize
\fi



\subparagraph{}
\textit{Après avoir proposé un paramétrage, indiquer dans quel sens tourne la roue si le moteur 28 (31) tourne dans le sens positif.}

\ifprof
\begin{corrige}
Voir figure précédente. Si le moteur tourne dans le sens positif, la roue tourne dans le sens négatif. 
\end{corrige}
\else
\fi

\subparagraph{}
\textit{Pour une vitesse de \SI{1500}{tr/min} en sortie de moteur, déterminer la vitesse de rotation de la roue. Le diamètre de la roue est de \SI{150}{mm}. Quelle est la vitesse du véhicule ? }
\ifprof
\begin{corrige}
Le rapport de réduction de la transmission est le suivant : 
$k=\dfrac{Z_{27} Z_{5} Z_{11} }{Z_{35} Z_{11} Z_{16}} = \dfrac{16\cdot 14}{84\cdot 75} =0,0355 $

La vitesse de rotation de la roue est donc de $\SI{53,33}{tr.min^{-1}}$ soit \SI{5,59}{rad.s^{-1}}. 
On en déduit la vitesse du véhicule : $5,59 \times 0,15 = \SI{0,84}{m.s^{-1}}\simeq \SI{3}{km.h^{-1}}$.

\end{corrige}
\else
\fi


\section*{Exercice 4 -- Transmission vis -- écrou}
\textit{D'après ressources Pole Chateaubriand -- Joliot-Curie.}
\setcounter{exo}{0}

\ifprof
\else

La conduite en ville nécessite des répétitions
fréquentes de la manœuvre d’embrayage /
débrayage. Pour améliorer le confort de conduite,
on peut substituer la force musculaire du
conducteur par une commande électrique de
l’embrayage. Dans ce cas, il devient nécessaire de
renseigner l’unité de contrôle électronique sur les
intentions du conducteur à partir d’un capteur de
position placé sur la pédale d’embrayage.
L’automatisation de la fonction embrayage
permet de corriger les éventuelles fausses
manœuvres du conducteur, d’assurer la fonction
anti-calage du moteur et de participer aux
fonctions d’anti-patinage et d’anti-blocage des
roues.

\begin{center}
\includegraphics[width=\linewidth]{images/rest_01}
\end{center}


En cas d’utilisation de la pédale, il faut recréer les sensations
au conducteur, c'est-à-dire une résistance mécanique proche
de celle d’une commande mécanique classique. Pour réaliser
ce système de retour d’effort la solution peut être passive (un
ressort, par exemple) ou utiliser un système actif (à l’aide d’un
actionneur électrique), objet de l’étude.
L’étude porte sur un démonstrateur de restituteur actif
d’effort à la pédale. Le démonstrateur permet de tester
différentes lois de restitution d’effort pour rechercher la plus
ergonomique. Le système contrôle, par l’intermédiaire d’un
piston, l’effort sur la tige de poussée de la pédale.


\begin{center}
\includegraphics[width=\linewidth]{images/rest_02}
\end{center}

Le schéma du restituteur actif est donné ci-dessous. Le pas de la vis est $p_v =\SI{10}{mm}$.
Le diamètre de la poulie 2 est le double de celui de la poulie 1. 


\begin{center}
\includegraphics[width=\linewidth]{images/rest_03}
\end{center}
\fi
\begin{obj}
Déterminer la première partie
de la loi entrée/sortie en vitesse du
système.
\end{obj}

\subparagraph{}
\textit{Sur le schéma cinématique, repasser chaque solide d’une couleur différente.}

\ifprof
\begin{corrige}
\end{corrige}
\else
\fi

\subparagraph{}
\textit{Compléter la chaîne d’énergie-puissance partielle en définissant les noms des transmetteurs et les grandeurs
d’entrée et de sortie cinématiques.}

\ifprof
\begin{corrige}
\end{corrige}
\else
\fi



\begin{center}
\includegraphics[width=\linewidth]{images/rest_04}
\end{center}

\subparagraph{}
\textit{Définir la loi entrée-sortie entre la vitesse de translation du piston 3 et la vitesse de rotation du moteur~1. }

\ifprof
\begin{corrige}
\end{corrige}
\else
\fi


\section*{5 -- Axe de machine-outil à commande numérique}
\textit{D'après ressources Pole Chateaubriand -- Joliot-Curie.}
\setcounter{exo}{0}

\ifprof
\else
L’usinage est une opération de transformation d’un produit par enlèvement de matière.
Cette opération est à la base de la fabrication de produits dans les industries mécaniques.
La génération d’une surface par enlèvement de matière est obtenue grâce à un outil muni
d’au moins une arête coupante. Les différentes formes de pièces sont obtenues par des
translations et des rotations de l'outil par rapport à la pièce.

\begin{center}
\includegraphics[width=.6\linewidth]{images/ugv_01}
\end{center}

On s’intéresse ici à l’axe Y qui met en mouvement le coulisseau 1,
sur lequel est fixée l’outil, par rapport au bâti 0. Le coulisseau 1 est mis en mouvement par un moteur
électrique qui délivre un couple moteur $C_m(t)$.

\begin{center}
\includegraphics[width=\linewidth]{images/ugv_02}
\end{center}

On note $p$ le pas de vis. 
\fi


\subparagraph{}
\textit{Définir la loi entrée-sortie entre la vitesse de translation du coulisseau et la vitesse de rotation du moteur. }
\ifprof
\begin{corrige}
\end{corrige}
\else
\fi


\section*{Exercice 6 -- Treuil de levage}
\textit{D'après ressources Pole Chateaubriand -- Joliot-Curie.}
\setcounter{exo}{0}
\ifprof
\else
On s’intéresse à un treuil dont la photo et le modèle cinématique sont donnés ci-dessous.
\begin{center}
\includegraphics[width=.5\linewidth]{images/treuil_01}
\end{center}

\begin{center}
\includegraphics[width=\linewidth]{images/treuil_02}
\end{center}

On note $Z_2$ le nombre de dents de la roue dentée de l'arbre 2. On note l'arbre intermédiaire 3 et $Z_{3a}$ et $Z_{3b}$ les nombres de dents de ses deux roues dentées. On note $R$ le rayon du tambour 4 sur lequel s’enroule sans glisser un câble et $Z_4$ le nombre de dents de sa roue dentée.

\fi

\subparagraph{}
\textit{Déterminer la relation entre $v_{51}$ la vitesse de déplacement de la charge par rapport au bâti et $\omega_{21}$ la vitesse de rotation du moteur.}
\ifprof
\begin{corrige}
\end{corrige}
\else
\fi

\ifprof
\end{multicols}
\else
\end{multicols}
\fi

\end{document}

\subparagraph{}
\textit{}

\ifprof
\begin{corrige}
\end{corrige}
\else
\fi

	