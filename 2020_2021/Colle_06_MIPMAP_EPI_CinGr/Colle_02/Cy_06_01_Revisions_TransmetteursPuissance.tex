\documentclass[10pt,fleqn]{article} % Default font size and left-justified equations
\usepackage[%
    pdftitle={Modélisation dynamique},
    pdfauthor={Xavier Pessoles}]{hyperref}

    
\input{style/new_style}
\input{style/macros_SII}
\usepackage{multicol}
\usepackage{siunitx}
%\usepackage{picins}
\fichetrue
%\fichefalse

\proftrue
\proffalse

\tdtrue
%\tdfalse

\courstrue
\coursfalse


\def\classe{\textsf{PSI$\star$ -- MP}}
\def\xxnumpartie{Cycle 06}
\def\xxpartie{Modéliser le comportement des systèmes mécaniques dans le but d'établir une loi de comportement ou de déterminer des actions mécaniques en utilisant les méthodes énergétiques}

\def\xxnumchapitre{Chapitre 1 \vspace{.2cm}}
\def\xxchapitre{\hspace{.12cm} Approche énergétique}

\def\discipline{Sciences \\Industrielles de \\ l'Ingénieur}
\def\xxtete{Sciences Industrielles de l'Ingénieur}




\def\xxtitreexo{Révisions -- Rapports de transmission}%Motorisation du moteur Haibike}
\def\xxsourceexo{%\hspace{.2cm} \footnotesize{Émilien Durif -- E3A PSI 2011}
}


\def\xxposongletx{2}
\def\xxposonglettext{1.45}
\def\xxposonglety{20}
%\def\xxonglet{Part. 1 -- Ch. 3}
\def\xxonglet{Cycle 06}

\def\xxactivite{Application}
\def\xxauteur{\textsl{X. Pessoles}}

\def\xxcompetences{%
\textsl{%
\textbf{Savoirs et compétences :}\\
%Les sources sont associées par un \emph{hacheur série}. La détermination des grandeurs électriques associées à ce montage permet de conclure vis à vis du cahier des charges.
%\noindent \textbf{Résoudre :} à partir des modèles retenus :
%\begin{itemize}[label=\ding{112},font=\color{ocre}] 
%\item choisir une méthode de résolution analytique, graphique, numérique;
%\item mettre en \oe{}uvre une méthode de résolution.
%\end{itemize}
%\begin{itemize}[label=\ding{112},font=\color{ocre}] 
%\item \textit{Rés -- C1.1 :} Loi entrée sortie géométrique et cinématique -- Fermeture géométrique.
%\end{itemize}
%
%\noindent \textit{Mod2 -- C4.1 :} Représentation par schéma bloc.
}}

\def\xxfigures{
%\includegraphics[width=.7\linewidth]{images/axe_y_photo}
}%figues de la page de garde


\def\xxpied{%
Cycle 06 -- Modélisation mécanique -- Énergétique\\% afin de valider leurs performances.\\
Chapitre 1 -- \xxactivite%
}

\setcounter{secnumdepth}{5}
%---------------------------------------------------------------------------

\usepackage{pgfplots}
\begin{document}
\def\pathfig{images}
%\chapterimage{png/Fond_Cin}
\input{style/new_pagegarde}
\vspace{4.5cm}
\pagestyle{fancy}
\thispagestyle{plain}

\def\columnseprulecolor{\color{ocre}}
\setlength{\columnseprule}{0.4pt} 

\def\pathfig{images}

\ifprof
%\begin{multicols}{2}
\else
\begin{multicols}{2}
\fi


\section*{Exercice 5 -- Train épicycloïdal -- Type 4 -- Poulie Redex}
\setcounter{exo}{0}

\textit{D'après ressources de Stéphane Genouël.}


\begin{obj}~\\
Vérifier les performances d'un réducteur.
\end{obj}

Il existe 2 grandes familles de poulies Redex H ou SR, selon la forme de l’arbre central.


\begin{center}
\includegraphics[width=.7\linewidth]{images/redex_01}
\end{center}



Le mouvement d’entrée est reçu par le boîtier tournant 5, entraîné par 5 courroies trapézoïdales 8, et guidé en rotation par rapport au bâti 18 à l’aide de deux roulements à billes 23 et 28. 
Les flasques 16 permettent le montage des organes intérieurs. Ils sont munis de joints d’étanchéité 22 et 29. 
Les trois axes 9, guidés en rotation par rapport au boîtier tournant 5 à l’aide de deux roulements à aiguilles 4 et 11, portent les trois satellites doubles 6-10.
Les liaisons encastrements entre les axes 9 et les satellites 6 et 10 sont assurées (élastiquement) par de la matière plastique injectée entre les axes et les pignons préalablement dentelés (voir coupe A-A et B-B). 
Les satellites 10 sont en prise avec le planétaire 24 (qui est en liaison encastrement avec le bâti 18 à l’aide d’un assemblage cannelé).
Les satellites 6 sont en prise avec le planétaire 31 (qui est en liaison encastrement avec l’arbre de sortie 32 à l’aide d’un assemblage cannelé). Cet arbre de sortie 32 est guidé en rotation par rapport au bâti 18 à l’aide de deux roulements à aiguilles 19 et 21.



\subparagraph{}
\textit{Déterminer littéralement, en fonction des nombres de dents, la loi E/S du système (c'est-à-dire le rapport de transmission).}


%
%\begin{center}
%\includegraphics[width=.8\textwidth]{images/fig_02}
%\end{center}
%
%

\begin{center}
\includegraphics[width=\linewidth]{images/redex_04}
\end{center}




\section{Benne de camion}

\setcounter{subparagraph}{0}
On se propose d'étudier le système qui assure l'ouverture d'une benne de camion de ramassage d'ordures.

\begin{minipage}[c]{.3\linewidth}
\begin{center}
\includegraphics[width=.95\textwidth]{images/fig3_1}\hfill
\end{center}
\end{minipage} \hfill
\begin{minipage}[c]{.65\linewidth}
\begin{center}
\includegraphics[width=.95\textwidth]{images/SysML/Exigences_Benne}
\end{center}
\end{minipage}

\begin{obj}
L'objectif est de vérifier que l'exigence 1.4.1 est vérifiée.
\end{obj}


Le schéma cinématique de la mise en mouvement du système est fourni sur la figure suivante. Un vérin impose le mouvement du système. Dans la position donnée, la vitesse de sortie de la tige 2 par rapport au corps du vérin 1 est de $0,1\; m/s$ (Echelle des vitesses : 3cm pour 0,1 m/s).


\subparagraph{}
\textit{Déterminer graphiquement avec les justifications utiles $\vectv{B}{3}{0}$ puis $\vectv{F}{3}{0}$}


\subparagraph{}
\textit{Déterminer $\omega(3/0)$ et conclure vis-à-vis du cahier des charges $(BO=6m)$.}

La benne est munie d'une porte 4 qui s'ouvre lorsque 3 s'incline.


\subparagraph{}
\textit{Déterminer graphiquement avec les justifications utiles $\vectv{C}{5}{0}$ et $\vectv{G}{5}{0}$.}


\subparagraph{}
\textit{Déterminer $\omega_{5/3}$.}



\end{multicols}



%\section*{Liaison encastrement démontable}
%\setcounter{exo}{0}
%
%\subparagraph{}
%\textit{Détailler l'architecture de la liaison encastrement démontable de votre choix.}
%
%
%\end{multicols}
%\begin{center}
%\includegraphics[width=.6\linewidth]{Dessin_02}
%\end{center}
\newpage
$$ \quad $$
\vspace{6cm}

\begin{center}
\includegraphics[width=.75\textwidth]{images/fig4}
\end{center}

\end{document}

\subparagraph{}
\textit{}

\ifprof
\begin{corrige}
\end{corrige}
\else
\fi

	